%-----------------------------------------------------------------------------
% Schriftgr��e, Layout, Papierformat, Art des Dokumentes
%-----------------------------------------------------------------------------
\documentclass[12pt,					% Grundschriftge
							 oneside,			% einseitiges Dokument
							 a4paper,			% Papiergr��e
							 halfparskip,		% Einzug bei einem Absatz
							 liststotoc,			% Verzeichniss (Abbildungen erc.) in das Inhaltverzeichnis
							 bibtotoc,			% Literaturverzeichnis ins Inhaltverzeichnis
							 fleqn,				% Mathematische Formeln linksb�ndig darstellen
							 pointlessnumbers]	% Punkt am Ende der Nummerierung des Inhaltsverzeichnisses 													entfernen
							 {scrreprt}

%-----------------------------------------------------------------------------
% Konstanten festlegen
%-----------------------------------------------------------------------------
\newcommand{\VerfasserC}{Christoph Werner}
\newcommand{\GeburtstagC}{16. Dezember 1998}
\newcommand{\GeburtsortC}{Dannenberg/ Elbe}
\newcommand{\EmailC}{j.prothmann@stud.hs-wismar.de}

\newcommand{\VerfasserJ}{Josef Prothmann}
\newcommand{\GeburtstagJ}{16. Dezember 1998}
\newcommand{\GeburtsortJ}{Dannenberg/ Elbe}
\newcommand{\EmailJ}{j.prothmann@stud.hs-wismar.de}

\newcommand{\Titel}{Data Mining Semesterprojekt Dokumentation}

\newcommand{\Betreuer}{Prof. Dr. rer. nat. Jürgen Cleve}


\newcommand{\blankpage}{
%	\newpage
%	\thispagestyle{empty}
%	\mbox{}
	\newpage
}

%-----------------------------------------------------------------------------
% verwendete Pakete
%-----------------------------------------------------------------------------
\usepackage[utf8]{inputenc}		% Zeichkodierung , Umlaute erlauben
\usepackage[T1]{fontenc}				% Wahl des Fonts, bzw. der Kodierung
\usepackage[english,ngerman]{babel}		% neue deutsche Rechtschreibung verwenden
\usepackage{graphicx}					% erm�glicht das Einbinden von Grafiken, sehr wichtig!
\usepackage{fancyhdr}					% f�r formatierte Kopf- und Fu�zeilen
\usepackage{setspace}					% Package zum Kontrollieren von Leerr�umen
\usepackage{subfigure}					% erweiterte Darstellung von Bildern
\usepackage{listings}					% M�glickeit zum Anzeigen von Quelltexten
\usepackage{color,moreverb}				% Farben
\usepackage{lmodern}					% bietet neuere Schriften, sieht besser aus im Acrobat Reader
\usepackage{amsmath,amssymb}			% erweiteter Formelsatz und zus�tzliche Mathe-Symbole
\usepackage{booktabs}					% professionelle, typographisch richtige Tabellen
\usepackage{cite}						% f�r LibTex
%\usepackage{shortvrb}					% f�r Quellcode mit \begin{verbatim}
\usepackage[binary-units=true]{siunitx}	% Darstellung von Si-Einheiten
%\usepackage{pdfpages}					% pdf-Seiten einbinden
\usepackage{enumitem}					% custom itemiziation





%-----------------------------------------------------------------------------
% Fu�notennummerierung nicht f�r jedes kapitel zur�cksetzen
%-----------------------------------------------------------------------------
\usepackage{chngcntr}
\counterwithout{footnote}{chapter}

%-----------------------------------------------------------------------------
% Einstellungen der Seitenr�nder
%-----------------------------------------------------------------------------
\usepackage[left=3cm,						% linker Rand
						right=3cm,			% rechter Rand
						top=1.5cm,			% oberer Rand
						bottom=1.5cm,		% unterer Rand
						includeheadfoot,	% bezieht die Kopf- und Fu�zeile mit ein
						bindingoffset=0cm]	% Bundsteg
						{geometry}


%-----------------------------------------------------------------------------
% Daten f�r die Titel des Artikels
%-----------------------------------------------------------------------------
\title{Semesterarbeit Data Mining}
\author{\VerfasserC, \Verfasserj}
\date{\today{}}

%-----------------------------------------------------------------------------
% Metadaten in pdf einf�gen
%-----------------------------------------------------------------------------
\usepackage[pdftex,
						pdfauthor={\VerfasserC,\VerfasserJ},									% Name des Autors
						pdftitle={\Titel},										% Name der Arbeit
						pdfcreator={MiKTeX, LaTeX with hyperref and KOMA-Script},	% Von was erzeugt
						pdfsubject={Semesterarbeit Data Mining},							% Was f�r eine Arbeit ist es
						pdfkeywords={\Titel},
						plainpages=false,
						hypertexnames=false,
						pdfpagelabels]{hyperref}

%-----------------------------------------------------------------------------
% Schriftarten anpassen
%-----------------------------------------------------------------------------
\setkomafont{sectioning}{\rmfamily\bfseries}			% Titelzeilen
\setkomafont{caption}{\small}							% Schrift f�r Caption
\setkomafont{captionlabel}{\sffamily\bfseries\small}	% Schrift f�r 'Abbildung'
\setkomafont{chapterentry}{\small\bfseries}				% Schrift f�r Inhaltsverzeichnis
\setkomafont{chapter}{\large\bfseries}					% Schrift f�r Kapitel
\setkomafont{section}{\normalsize}						% Schrift f�r Section
\setkomafont{subsection}{\normalsize}					% Schrift f�r Subsection



%-----------------------------------------------------------------------------
% Farbe f�r Links in PDF-Dokumenten definieren
%-----------------------------------------------------------------------------
\definecolor{LinkColor}{rgb}{0,0,0}				% Festlegen einer neuen Farbe

\hypersetup{colorlinks=true,					% farbliche Links
						breaklinks=true,		% Zeilenumbruch erlauben
						linkcolor=black,		% Farbe f�r interne Links
						citecolor=black,		% Farbe f�r Links zum Literaturverzeichnis
						filecolor=LinkColor,	% Farbe f�r externe Dateilinks
						menucolor=LinkColor,	%
						urlcolor=LinkColor}		% Farbe f�r externe Links
						



				
%-----------------------------------------------------------------------------
% Kopf- und Fusszeile bestimmen
%-----------------------------------------------------------------------------
\pagestyle{fancy}	
\fancyhf{}												% alle Felder l�schen
\fancypagestyle{plain}{}

% Kopfzeile rechts bzw. au�en
\fancyhead[R]{\nouppercase{\leftmark}}
% Linie oben
\renewcommand{\headrulewidth}{0.5pt}
% Fu�zeile rechts bzw. au�en
\fancyfoot[R]{\thepage}
%-----------------------------------------------------------------------------

%-----------------------------------------------------------------------------
% Begin des Dokuments
%-----------------------------------------------------------------------------

\begin{document} 						% Beginn des Dokumentes

	\renewcommand\lstlistingname{Code}
	\renewcommand\lstlistlistingname{Codeverzeichnis}
	
	%% Titel
	\begin{titlepage}
		\setlength\headsep{-5mm}
		\begin{figure}[!h]
			\begin{minipage}{0.8\textwidth}
				\textbf{Hochschule Wismar} \\
				University of Applied Sciences \\
				Technology, Business and Design \\
				Fakultät für Ingenieurwissenschaften, Bereich EuI \\
			\rule{\textwidth}{0.5pt}
			\end{minipage}
			\begin{minipage}[r]{0.1\textwidth}
				\begin{flushright}
					\includegraphics[height=6\baselineskip]{pictures/HS-Wismar_Logo-FIW_2010-01.jpg}
				\end{flushright}
			\end{minipage}
		\end{figure}
		\vspace*{6cm}
		\begin{center}
			\Huge
			\textbf{Semesterarbeit} \\
			\vspace{2cm}
			\large \Titel
			\begin{table*}[b]
				\begin{tabular}{rl}
					%Gedruckt am: & \today \\
					
					Eingereicht am: &\today \\
					\\
					  & \VerfasserC \\ 
					& geboren am \GeburtstagC \\ 
					%& in \GeburtsortJ \\
					& Email: \EmailC \\
					\\
					 & \VerfasserJ \\ 
					& geboren am \GeburtstagJ \\ 
					%& in \GeburtsortJ \\
					& Email: \EmailJ \\
					\\

					Dozent: & \Betreuer \\

				\end{tabular}
			\end{table*}
		\end{center}
	\end{titlepage}

	\onehalfspacing 					% 1 1/2-zeilig (package 'setspace')
	
	%\blankpage	%leeres Blatt zwischen Deckblatt und Inhaltsverzeichnis	

	%-----------------------------------------------------------------------------
	% Inhaltsverzeichnis
	%-----------------------------------------------------------------------------	
	\pdfbookmark[1]{Inhaltsverzeichnis}{toc}	% Inhaltsverzeichnis zu den Lesezeichen hinzuf�gen
	%\singlespacing 						% 1-zeilig
	
	%\onehalfspacing 					% 1 1/2-zeilig (package 'setspace')
\section*{Abstrakt}

\section*{Abstract}


\newpage

\tableofcontents 					% Inhaltverzeichnis einf�gen
	%-----------------------------------------------------------------------------
	% Hauptteil
	%-----------------------------------------------------------------------------	
	
\chapter{Grundlagen}

Bibtexkey\cite{Cleve2020}
\section{Datenvorverarbeitung}
\section{Entscheidungsbäume}

\section{Clusteranalysen}

\chapter{KNIME Implementierung}
	%-----------------------------------------------------------------------------
	% Literaturverzeichnis einf�gen, 
	% Nutzung der BibTeX-Technologie --> literatur.bib 
	%-----------------------------------------------------------------------------

	
	\bibliographystyle{unsrtdin}		%  Stil des Literaturverzeichnisses (hier nach DIN 1505)
	\bibliography{literature}			% gibt Datei mit der Literatur an
	
	\nocite{*}						% damit alle in der DB enthaltende Eintr�ge bearbeitet werden
	

	%-----------------------------------------------------------------------------
	% Verzeichnisse
	%-----------------------------------------------------------------------------
	\listoffigures						% Bildverzeichnis einf�gen

	%-----------------------------------------------------------------------------
	% Anhang
	%-----------------------------------------------------------------------------	
	\appendix
	% Auch hier sind Gliederungen aller \chapter, \section
	

	%-----------------------------------------------------------------------------
	% Selbstst�ndigkeitserkl�rung
	%-----------------------------------------------------------------------------	
%	\chapter*{Selbstst\"andigkeitserkl\"arung}
%	\addcontentsline{toc}{chapter}{Selbstst\"andigkeitserkl\"arung}
%	\rhead{Selbstst\"andigkeitserkl\"arung} % rechts oben in der Kopfzeile Chapter darstellen
%	Hiermit erkl\"are ich, dass ich die hier vorliegende Arbeit selbstst\"andig,
%	ohne unerlaubte fremde Hilfe und nur unter Verwendung der aufgef\"uhrten
%	Hilfsmittel angefertigt haben.
%
%	\begin{tabular}{p{10cm}p{13cm}}
%		\\
  %		\\
  	%	\\
 % 		\\
 % 		Wismar, den \today \\
 % 		---------------------------------------  & ------------------------ \\
 % 		Ort, Datum & Unterschrift
%	\end{tabular}
	


\end{document}							% Ende des Dokuments
%-----------------------------------------------------------------------------
